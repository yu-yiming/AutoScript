\begin{introduction}
    \item 简单语句
    \item 语句块
    \item \lstinline!break! 语句
    \item \lstinline!continue! 语句
    \item 声明语句
    \item 模式声明
    \item \lstinline!else! 子句
    \item \lstinline!then! 子句
    \item \lstinline!return! 语句
    \item \lstinline!throw! 语句
    \item \lstinline!assert! 语句
    \item \lstinline!assume! 语句
    \item \lstinline!try! 语句
\end{introduction}

\section{语句}

\textbf{语句(Statement)}代表了 AutoScript 中的一个单独操作 \footnote{注意区分单独操作和\emph{原子操作}:单独操作是指在语义上是一个操作,而原子操作是指其可以被转译为一个单独的汇编指令。}。语句之间大多通过 \lstinline!;! 进行分隔,少部分(以块结构结尾的)语句不需要加分号。

\subsection{简单语句}

简单语句顾名思义是结构非常简单的语句,它包括两种情况。

\begin{grammar}[简单语句] \label{grm:simple-statement}
\begin{enumerate}
	\item \lstinline!;!
	\item \lstinline![expr];!
\end{enumerate}
\end{grammar}

可以看到,简单语句包含了空语句和求值语句,前者不做任何事,后者会对表达式进行求值(除非它是一个 \lstinline!lazy! 表达式,此时也不会做任何事)。

\subsection{块语句和跳转语句}

\textbf{块语句(Block Statement)},也被称为\textbf{语句块(Block)} 是由一对大括号组成的结构,其中包含了零到多个语句。语句块定义了一个\emph{作用域}。回忆第二章中我们曾经介绍过,不同作用域中允许声明相同名字的变量。我们可以用 \lstinline!@name! 属性为作用域命名,它会出现在语句块中变量的名字中。

\begin{lstlisting}
const main = func () -> do @name(body) {
    @name(inner) {
        auto x = 42;
        print nameof(x);		// 输出 ::main::body::inner::x
    }
};
\end{lstlisting}

我们可以用 \lstinline!break! 语句来跳出某个语句块。

\begin{grammar}[\texttt{break} 语句 \texttt{[break-stmt]}] \label{grm:break-statement}
\begin{enumerate}
	\item \lstinline!break! \texttt{[scope-id]} \lstinline!;!
	\item \lstinline!break;!
\end{enumerate}
\end{grammar}

第一种情况跳出给定的作用域,而第二种默认跳出当前作用域。

\begin{lstlisting}
const main = func () -> do {
    @name(outer) {
        scan x : Int;
        if (x > 0) {
            break outer;
        }
    }
};
\end{lstlisting}

只有局部作用域中允许使用 \lstinline!break! 语句。要注意,\lstinline!do! 和 \lstinline!monad! 表达式后的语句块不能通过 \lstinline!break! 语句跳出。 \\

特别地,我们可以用 \lstinline!last! 表示上一个作用域。这里的 \lstinline!last! 是一个上下文关键字,它只在特定语句中有作用。

此外,可以使用 \lstinline!continue! 语句跳回作用域的开始。

\begin{grammar}[\texttt{continue} 语句 \texttt{[continue-stmt]}] \label{grm:continue-statement}
\begin{enumerate}
	\item \lstinline!continue! \texttt{[scope-id]} \lstinline!;!
	\item \lstinline!continue;!
\end{enumerate}
\end{grammar}

第一种情况跳到给定的作用域开始,而第二种默认跳到最内层的作用域开始。

\begin{lstlisting}
const main = func () -> do @name(body) {
    scan x : Int;
    @name(inner) {
        scan y : Int;
        if (x * y > 0) {
            continue inner;
        }
        if (x * y < 0) {
            continue body;
        }
    }
};
\end{lstlisting}

可以看到,\lstinline!continue! 语句允许用在 \lstinline!do! 表达式的语句块上。 \\

最后值得一提的是,块语句中的最后一个语句允许不添加分号;这和表达式中在最后可以添加逗号同理。

\begin{lstlisting}
const main = func () -> do { return 42 };	// 这里没有在 return 语句中使用分号。
\end{lstlisting}

\subsection{声明语句}

本节让我们给出声明语句的完整语法。

\begin{grammar}[声明语句] \label{grm:declaration-statement}
\begin{enumerate}
    \item \lstinline![storage-spec] [id] :! \texttt{[type-spec]} \lstinline!;!
    \item \texttt{[pattern]} \lstinline!=! \texttt{[expr]} \lstinline!;!
\end{enumerate}
\end{grammar}

可以将上面的语法和语法 \ref{grm:variable-declaration-basics} 比较。第一种情况是一个\emph{提前声明},提示编译器某个名字拥有的类型。第二种情况下,变量声明包含一个\textbf{模式(Pattern)}和初始化表达式。这里,\lstinline![pattern]! 是一个\textbf{模式声明(Pattern Declaration)},其中包含了一个特定结构,且可能会引入一些名字(在声明语句中,总会引入一个名字)。\lstinline![expr]! 是一个表达式,它的值这里被称为待匹配实体。有关模式声明,请参考下面的一些例子:

\begin{lstlisting}
const main = func () -> do {
    auto tup = (1, 2);
    auto (m, n) = tup;      // 这里使用元组模式声明 (m, n)
    print m;                // 输出 1
    print n;                // 输出 2
};
\end{lstlisting}

我们可以在模式声明中再次使用模式声明,这也就是模式的嵌套:

\begin{lstlisting}
const main = func () -> do {
    auto tup = (1, 2);
    auto (m : mut, n) = tup;// 这里的 m : mut 是一个原子模式,它嵌在元组模式 (m : mut, n) 中
    print typeof(m);        // 输出 mut Int
    print typeof(n);        // 输出 Int
};
\end{lstlisting}

注意到 \lstinline!m : mut! 中并没有将 \lstinline!m! 的类型完全写出来,这是因为我们可以通过 \lstinline!tup! 来推断它的类型(并在最外层加上 \lstinline!mut! 修饰)。如果要得到某个实体的引用,则可以通过下面的方式:

\begin{lstlisting}
const main = func () -> do {
    auto tup = mut (1, 2);
    auto (ref m, n) = ref tup;  // 这里的 ref m 是一个引用模式声明
    print typeof(m);            // 输出 Int
    print typeof(n);            // 输出 ref Int
};
\end{lstlisting}

这里我们看到,元组的引用类型可以被匹配为元组模式,其中的每个元素类型都是原来元组中元素的引用类型。 \\

这里有必要说明出现在 模式声明中的 \lstinline!ref! 和作为类型限定符的 \lstinline!mut! 的区别。模式声明会匹配对象进行强匹配:被匹配的值需要完全满足模式声明中的结构,但类型限定符允许初始化值被隐式类型转换为声明的类型。

\begin{lstlisting}
const main = func () -> do {
    auto x = 42;
    auto y = mut 42;
    auto mut z = x;         // 编译错误,mut y 无法匹配 x,后者不是可变类型
    auto mut z = y;         // 没问题
    print typeof(z);        // 输出 Int,因为 mut y 作为模式声明整体才是 mut Int
};
\end{lstlisting}

若非必要,并不推荐将 \lstinline!ref! 以外的修饰符用于模式声明中,因为它会造成迷惑 \footnote{确实,明明声明为 \lstinline!mut x!,但 \lstinline!x! 本身一定不是可变类型。初见时可能非常反直觉,但对 AutoScript 中的模式熟悉之后,这会变得自然。}。此后我们会更深入地介绍 AutoScript 的模式系统。

\subsection{条件语句}

我们在第一章中已经介绍过 \lstinline!if! 语句的用法。不过,光有 \lstinline!if! 语句本身似乎表达力不够充足:我们已经判断过的表达式,此后应该不需要再反过来判断一遍才对。此时可以使用 \lstinline!else! \emph{子句}。

\begin{grammar}[\texttt{else} 子句]
    \lstinline![stmt] else [block]!
\end{grammar}

\lstinline!else! 子句和 \lstinline!if! 语句一起使用时,当且仅当 \lstinline!if! 语句的表达式判断为假时会执行 \lstinline!else! 后的语句块。

\begin{lstlisting}
const main = func () -> do {
    scan x : Int;
    if (x < 0) {
        print "Negative";
    }
    else {                      // 以 x < 0 为假为前提
        print "Non negative";
    }
};
\end{lstlisting}

特别地,\lstinline!else! 子句中不仅可以放一个语句块,也可以放一个 \lstinline!if! 语句,因此我们可以连续写多个互斥的 \lstinline!if! 语句来筛选情况。

\begin{lstlisting}
const main = func () -> do {
    scan x : Int;
    if (x < 0) {
        print "Too small";
    }
    else if (x < 20) {
        print "A bit small";
    }
    else if (x < 80) {
        print "Good";
        return;
    }
    else if (x < 100) {
        print "A bit large";
    }
    else {
        print "Too large";
    }
    continue;
};
\end{lstlisting}

\subsection{循环语句}

从本章新介绍的 \lstinline!continue! 出发,我们可以实现一个最原始的循环语句:

\begin{lstlisting}
const main = func () -> do {
    {
        // 做一些事情
        continue;
    }
};
\end{lstlisting}

实际上,\lstinline!while! 语句可以看作是上面这种写法的语法糖:

\begin{lstlisting}
const main = func () -> do {
    scan b : Bool;
    while (b) {
        // 做一些事情
    }

    // 上面的代码和下面的等价
    {
        // 做一些事情
        continue if (b);
    }
};
\end{lstlisting}

显然 \lstinline!while! 语句有更好的可读性。以上面的等价代码为启示,我们也可以猜到 \lstinline!break! 和 \lstinline!continue! 语句在 \lstinline!while! 代码块中的作用。值得特别一提的是 \lstinline!else! 子句在 \lstinline!while! 语句中的作用。如果 \lstinline!while! 语句由于没有满足表达式条件退出,则会进入后面的 \lstinline!else! 子句。

\begin{lstlisting}
const main = func () -> do {
    scan x : Int;
    while (x > 0) {
        print "Good";
        if (x == 42) {
            print "Pretty good";
            break;
        }
        print "Try again";
        scan x;
    }
    else {
        print "Bad";
    }
};
\end{lstlisting}

上面的程序中,如果某次输入了负数,则会输出 \lstinline!Bad! 然后程序结束。否则则会反复要求输入整数,直到输入 \lstinline!42! 后通过 \lstinline!break! 语句退出循环。注意通过 \lstinline!break! 语句退出时不会进入 \lstinline!else! 子句。 \\

如果想要让 \lstinline!break! 出来的控制流进行特殊操作,可以使用 \lstinline!then! 子句。

\begin{grammar}[\texttt{then} 子句]
    \lstinline![stmt] then [block]!
\end{grammar}

\lstinline!then! 子句和 \lstinline!else! 子句可以一起使用。在同一个语句中,它们的顺序并不重要,但只能出现一次。注意,虽然可以在这两个子句中使用 \lstinline!break! 语句,但它并不能通过再添加一个 \lstinline!then! 语句来接管控制流。

\begin{lstlisting}
const main = func () -> do {
    scan x : Int;
    while (x >= 0) {
        if (x == 0) {
            print "Zero";
            break;
        }
        if (x == 42) {
            print "Secret";
            break;
        }
        scan x;
    }
    then {
        print "You've passed!";
    }
    else {
        print "You've failed";
    }
};
\end{lstlisting}

\subsection{遍历语句}

AutoScript 中的遍历语句 \lstinline!for! 和其它语言(如 C++)的含义有一定差别。它的本质和循环语句不同。

\begin{lstlisting}
const main = func () -> do {
    for (i from range(0, 10)) {			// 这里 i 的类型是 Int
        print i;                        // 从 0 输出到 9
    }
    auto maybe_int = Some 42;			// maybe_int 的类型是 Option[Int]
    for (i from maybe_int) {
        print i;                        // 输出 42
    }
};
\end{lstlisting}

上面用到了 \lstinline!Prelude! 中定义的 \lstinline!range! 函数,其返回一个左闭右开的区间。\lstinline!Just 42! 构建了一个 \lstinline!Option! 类型。可以看到,\lstinline!for! 语句可以将一些类型中的元素取出,这些类型并不限定是一个\emph{范围}。 \\

\lstinline!for! 语句中也可以使用 \lstinline!else! 和 \lstinline!then! 子句。当 \lstinline!for! 遍历的对象为空时(或遍历到最后),会进入 \lstinline!else! 子句。否则(通过 \lstinline!break! 语句跳出时)进入 \lstinline!then! 子句。

\begin{lstlisting}
const main = func () -> do {
	scan arr : Array(Int, 3);
	print "( ";
    for (e from arr) {
        print e;
    }
    then {
    	print ")";
    }
    else {
        print "The array is empty";
    }
};
\end{lstlisting}

\subsection{返回语句}

AutoScript 中,返回语句用来跳出 \lstinline!do! 或 \lstinline!monad! 表达式(相当于 \lstinline!break! 语句之于普通的语句块,不过返回语句需要返回一个值)。

\begin{grammar}[\texttt{return} 语句] \label{grm:return-statement}
    (1)\ \lstinline!return [expr];! \\
    (2)\ \lstinline!return;!
\end{grammar}

\lstinline!return! 语句将一个表达式的值作为返回值。如果没有给出这个表达式,则返回空对象 \lstinline!()!。我们已经使用它很多次了。在 \lstinline!do! 表达式的末尾会默认插入一个 \lstinline!return! 语句。

\begin{lstlisting}
const foo = func () -> do {
	scan x : Int;
	if (x == 42) {
		return 0;
	}
	// 这里会默认插入 `return;`,但这和此前推导的 do 表达式类型不同,因此会产生编译错误。
};
\end{lstlisting}

\begin{grammar}[\texttt{throw} 语句] \label{grm:throw-statement}
    (1)\ \lstinline!throw [expr];! \\
    (2)\ \lstinline!throw;!
\end{grammar}

\lstinline!throw! 语句将构成返回表达式的异常类型。如果当前控制流捕获了一个异常(通过 \lstinline!catch! 子句,我们很快就会介绍),则可以由 \lstinline!throw;! 再次抛出这个异常 \footnote{这里的语法设计是为了贴合 C++ 中的惯用。从其它两种返回语句的语法类推,这里 \lstinline!throw! 应该指 \lstinline!throw ();! 才对,但实际并非如此。}。

\begin{grammar}[\texttt{yield} 语句] \label{grm:yield-statement}
    (1)\ \lstinline!yield [expr];! \\
    (2)\ \lstinline!yield;!
\end{grammar}

\lstinline!yield! 语句用在\emph{协程}中,其暂停当前的协程并返回一个值。	

\begin{lstlisting}
const ints = func () -> do {
	auto i = mut 0;
	yield i;
	i <- i + 1;
	continue;
};

const main = func () -> do {
	for (i from ints()) {
		print i;					// 无限地打印从自然数(0, 1, 2, ...)。
	}
};
\end{lstlisting}

我们会在并发章节中详细介绍 \lstinline!yield! 语句。

\subsection{输入输出语句}

AutoScript 利用 \lstinline!print! 语句进行输出,我们已经多次使用了。

\begin{grammar}[\texttt{print} 语句] \label{grm:print-statement}
    \lstinline!print [expr];!
\end{grammar}

默认情况下,\lstinline!print! 会打印在命令行上。如果想要手动选择输出的目标,可以使用属性 \lstinline!@stream! 属性,其中可以放入一个已经打开的文件流。对于 \lstinline!print! 语句,会让其内容在给定的文件流中输出。

\begin{lstlisting}
import System;

const main = func () -> do {
    auto file = System.open("output.txt", "w");
    @stream(file)
    print "Hello, world!";
};
\end{lstlisting}

此外,也可以用 \lstinline!@endwith! 来指定 \lstinline!print! 结尾的字符(默认为 \lstinline!\n!)。 \\

\lstinline!scan! 语句则用来输入。

\begin{grammar}[\texttt{scan} 语句] \label{grm:scan-statement}
    (1)\ \lstinline!scan [expr];! \\
    (2)\ \lstinline!scan [decl-stmt]!
\end{grammar}

和 \lstinline!print! 语句类似地,我们可以通过 \lstinline!@stream! 属性来指定一个输入流。

\begin{lstlisting}
import System;

const main = func () -> do {
    auto file = System.open("input.txt", "r");
    @stream(file)
    scan text : String;
};
\end{lstlisting}

\subsection{优化提示语句}

AutoScript 提供了一系列语句来为编译器进行优化提示。其中最常见的是用于断言的 \lstinline!assert! 语句。

\begin{grammar}[\texttt{assert} 语句] \label{grm:assert-statement}
    \lstinline!assert [expr];! 
\end{grammar}

若 \lstinline!assert! 后的表达式在编译期判断为假,则编译器可以中止编译并输出断言错误。如果这个表达式无法编译期求值,则会生成一段代码,在运行期判断为假时抛出断言异常。

\begin{lstlisting}
const main = func () -> do {
    assert 1 == 0;          // 编译错误,断言失败
    scan x : Int;
    assert x == 42;         // 编译期无法判断

    // 编译器会生成下面的代码
    if (not (x == 42)) {
        throw AssertionError("Expected x == 42");
    }
};
\end{lstlisting}

可以使用 \lstinline!@alert! 属性让 \lstinline!assert! 失败时调用自定义的函数来输出错误信息。这常用于编译器的 Debug 提示。

\begin{lstlisting}
const message = func () -> print "Use common sense!";

const main = func () -> do {
    @alert(message)
    assert 1 == 0;
};
\end{lstlisting}

注意 \lstinline!@alert! 中传入的函数必须不接入任何参数。 \\

比 \lstinline!assert! 语句“温和”一些的是 \lstinline!assume! 语句。

\begin{grammar}[\texttt{assume} 语句] \label{grm:assume-statement}
    \lstinline!assume [expr];!
\end{grammar}

编译期会将表达式 \lstinline![expr]! 当作优化提示。如果实际情况并非如此可能会出现未定义行为。

\begin{lstlisting}
const main = func () -> do {
    scan x : Int;
    assume x > 0;
    if (x == 0) {
        print "Impossible...?";
    }
};
\end{lstlisting}

上面的例子中,编译期通过 \lstinline!assume! 的表达式知晓了 \lstinline!x! 理应大于零,因此在分支预测时会默认跳过。如果在运行期我们输入了一个非正数,那么就是一个未定义行为。 \\

一些编译期可能无法通过的语句可以用预演语句 \lstinline!try! 来尝试。

\begin{grammar}[\texttt{try} 语句] \label{grm:try-statement}
    \lstinline!try [stmt]!
\end{grammar}

如果编译期就发现这个语句不可能执行,则编译器会将其消除。如果无法编译期判断(比如使用了动态类型),则会在运行期检查并选择性跳过。

\begin{lstlisting}
const main = func () -> do {
    auto x = 42;
    try x <- 20;            // 预演失败,因为 Int 不是可变类型
    auto y = dyn 42;
    try y.val <- 24;		// 运行时预演失败,因为 y 没有 data 这个动态成员
};
\end{lstlisting}

可以使用 \lstinline!@alert! 属性让 \lstinline!try! 语句预演失败时调用一个函数。

\begin{lstlisting}
const message = func () -> print "Check mutability!";

const main = func () -> do {
    auto x = 42;
    @alert(message)
    try x <- 20;
};
\end{lstlisting}

最后还有用于类型约束缓存的 \lstinline!implement! 语句,让我们在模版的章节中再详细介绍。

\section{子句}

此前我们已经见过 \lstinline!if!、\lstinline!else!、\lstinline!then! 等子句了。本节中让我们介绍一些其它常用的子句。

\subsection{where 子句}

在任何语句中都可以使用 \lstinline!where! 子句。

\begin{grammar}[\texttt{where} 子句] \label{grm:where-clause}
    \lstinline!where [block]!
\end{grammar}

在 \lstinline!where! 子句中的语句块中定义的任何变量都会引入到当前语句中。

\begin{lstlisting}
const main = func () -> do {
    auto fib_10 = fib(10) where {
        auto fib : Int -> Int;
        auto fib = func (x : Int) -> do {
            if (x <= 1) {
                return x;
            }
            return fib(x-1) + fib(x-2);
        };
    };
};
\end{lstlisting}

同一个语句中只允许出现一个 \lstinline!where! 子句。

\subsection{with 子句}

可以通过 \lstinline!with! 子句将一些自动存储期的变量引入到内部作用域,从而提前释放这个变量。

\begin{grammar}[\texttt{with} 子句] \label{grm:with-clause}
    \lstinline!with [id-tuple]!
\end{grammar}

\begin{lstlisting}
const main = func () -> do {
    auto x = 42;
    auto str = "abc";
    {
        auto x = 24;    // 错误,因为 x 被引入了当前作用域
    } with (x, str);
};
\end{lstlisting}

\subsection{requires 子句}

在声明语句中都可以使用 \lstinline!requires! 子句,它要求当前声明的对象拥有特定的性质,否则产生编译错误。

\begin{grammar}[\texttt{requires} 子句] \label{grm:requires-clause}
	\lstinline!requires [expr]!
\end{grammar}

\begin{lstlisting}
const main = func () -> do {
	auto x = 42 requires True;				// 没问题
	auto y = 24 requires sizeof(y) == 0;	// 编译错误
};
\end{lstlisting}




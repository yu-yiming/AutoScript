\begin{introduction}
	\item \texttt{asc}
	\item 编译到 C++
\end{introduction}

AutoScript 的编译器称为 \texttt{asc},即\emph{AutoScript Compiler}。它可以将一系列 AutoScript 文件编译成一或多个目标文件。本章中,我们将介绍 \texttt{asc} 编译原理以及编译选项。

\section{编译目标}

如果要指定目标语言,可以使用 \texttt{--target} 选项(\texttt{-t}) \texttt{asc} 支持的目标如下:

\begin{itemize}
	\item \texttt{C++}:AutoScript 和 C++ 有较强的交互性,其中一点就表现为它可以编译成标准 C++。可选的版本有 C++11、C++14 和 C++17。可以使用参数 \texttt{cpp} 来选择编译到 C++,其中 \texttt{11}、\texttt{14}、\texttt{17} 分别表示了不同版本,\texttt{14} 为默认。
	\item \texttt{LLVM}:AutoScript 可以编译为 LLVM,随后进一步编译成各个平台的机器码。可以使用 \texttt{llvm} 来选择编译到 LLVM。
	\item \texttt{ASBC}:即\emph{AutoScript Byte Code},AutoScript 可以编译成字节码后运行在 ASVM(AutoScript Virtual Machine)上。可以使用 \texttt{asbc} 来选择编译到 ASBC。
\end{itemize}

默认情况下,\texttt{asc} 会将文件编译到 LLVM 上,此时会直接生成一个可执行文件。

\begin{lstlisting}[language=Python]
asc main.as --target cpp 17
\end{lstlisting}

上面这个指令会将 \texttt{main.as} 编译为两个 C++ 文件 \texttt{main.hpp} 和 \texttt{main.cpp},其中包含了符合 C++17 标准的代码。


\section{编译到 C++}

本节让我们详细介绍 AutoScript 编译到 C++ 的思路和一些选项。

\subsection{实体}

AutoScript 中的实体涵盖了对象、函数、结构等范畴,但 C++ 中只有对象是一等公民,这就导致编译过程中必须做一些变形。

\subsubsection{对象}

对象是最好处理的实体范畴,因为我们可以直接用 C++ 变量来表示一个实体。首先是内置类型:

\begin{itemize}
	\item \lstinline!Int!:会被转换成 C++ 的 \cpp{long long} 类型。
	\item \lstinline!Real!:会被转换成 C++ 的 \cpp{double} 类型。
	\item \lstinline!String!:会被转换成 C++ 的 \cpp{builtin_string} 类型。
	\item \lstinline!Byte!:会被转换成 C++ 的 \cpp{builtin_byte} 类型。
	\item \lstinline!Void! 会被转换成 C++ 的 \cpp{builtin_void} 类型。
\end{itemize}

上面出现的 \lstinline!builtin_string!、\lstinline!builtin_byte! 和 \lstinline!builtin_void! 都是 C++ 中的自定义类型。最后一个内置类型 \lstinline!Type! 因为本质描述了一个类型,我们会在类型的子小节中谈到它。 \\

\lstinline!builtin_string! 支持基本的字符串操作。这里不进行详述。\lstinline!builtin_byte! 在 C++17 后可以使用标准库中的 \lstinline!std::byte! 类型(即定义为其别名),否则会使用自定义的类型。

\begin{lstlisting}[language=C++]
#if __cplusplus >= 201703L
#include <cstddef>
using builtin_byte = std::byte;
#else
class builtin_byte { /* 具体实现 */ };
#endif
\end{lstlisting}

\lstinline!builtin_void! 是一个很简单的类型,即一个空结构。 \\

现在从基本类型出发,我们需要将复合类型对象也映射到 C++ 的对象上。首先是\emph{元组}。我们使用 C++11 标准库中新增的 \lstinline!std::tuple! 类模版。注意到元组可以进行 \lstinline!size!、\lstinline!types! 和调用操作,它们可以通过下面的 C++ 函数实现:

\begin{lstlisting}[language=C++]
template<typename... Ts> constexpr std::size_t 
builtin_tuple_size(std::tuple<Ts...> const& tup) {
	return sizeof...(Ts);
}

template<typename... Ts> constexpr std::array<builtin_type, sizeof...(Ts)> 
builtin_tuple_types(std::tuple<Ts...> const& tup) {
	return { type_index(Ts)... };
}

template<std::size_t I, typename... Ts> constexpr auto 
builtin_tuple_get(std::tuple<Ts...>& tup) -> decltype(std::get<I>(tup)) {
	return std::get<I>(tup);
}
\end{lstlisting}

\emph{选择}类型则可以使用 C++17 标准库中的 \lstinline!std::variant! 类模版,不过对于之前的版本只能使用自定义的类型。选择类型需要强制类型转换为特定标签类型时,会调用 \lstinline!builtin_choice_convert! 函数模版。

\begin{lstlisting}[language=C++]
template<typename T, typename... Ts> T&
builtin_choice_convert(std::variant<Ts...>& var) {
	return std::get<T>(var);
}
\end{lstlisting}

\emph{延展}类型在 C++ 标准库中没有相应的实现,因此我们需要定义一个 \lstinline!builtin_extension! 来表示这个复合类型。

\begin{lstlisting}[language=C++]
template<typename... Ts>
class builtin_extension : public Ts... { /* 具体实现 */ };
\end{lstlisting}


\subsubsection{函数}

AutoScript 中的函数是一等公民:它们可以作为函数的参数或返回值传递,且可以声明在任何作用域。显然 C++ 中的函数局限性很大,但我们可以利用\emph{仿函数},即重载了 \lstinline!operator ()! 的对象。唯一的问题是后者拥有不稳定的类型,这会让常规的变量赋值操作变得不可能。为此,我们将函数定义为整数类型,其通过哈希表映射到一个进行过类型擦除的对象上。

\begin{lstlisting}[language=C++]
class builtin_function {
public:
	class function_base {
		virtual void* operator ()(Args... args) = 0;
	};
	

private:
	std::any m_invokable;
	signature m_sig;
};
\end{lstlisting}


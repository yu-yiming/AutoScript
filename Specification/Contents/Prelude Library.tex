\begin{introduction}
    \item \lstinline!Prelude.Bool!
\end{introduction}

\section{Prelude 库概览}

\lstinline!Prelude! 是所有 AutoScript 文件默认导入且\emph{内联}的库,其中包含了许多必要的组件。如果想要去除某些组件,可以显式导入 \lstinline!Prelude! 并用 \lstinline!excluding! 关键字指定特定的名字。

\begin{lstlisting}
@inline
import Prelude excluding (Bool);        // 禁用 Prelude.Bool
const main = func () -> do {
    auto b = True;          // 错误,True 未被导入
};
\end{lstlisting}

我们不推荐禁用任何 \lstinline!Prelude! 中的组件,这可能会导致部分的语言特性无法正常使用。

\section{内置类型扩展}

\subsection{Bool 类型}

\lstinline!Prelude.Bool! 是一个抽象数据类型,它包含 \lstinline!True! 和 \lstinline!False! 两个标签。

\begin{lstlisting}
export const Bool = data True | False;
\end{lstlisting}

AutoScript 中所有表达式判断都依赖于 \lstinline!Prelude.Bool!。某种意义上,它本应是一个内置的类型。但由于它是一个典型的抽象数据类型,且为了减少语言中不必要的关键字,我将其加入到 \lstinline!Prelude! 库中。 \\

\lstinline!Prelude.Bool! 实现了 \lstinline!Printable! 和 \lstinline!Scannable! 类族,因此我们可以用 \lstinline!print! 和 \lstinline!scan! 语句进行输入输出操作。

\begin{lstlisting}
export const serialize = func (b : Bool) -> b and "True" or "False";
export const deserialize = func (s : String) -> s == "False" and False or True;

implement Printable Bool;		// 声明 Bool 实现了 Printable 类族
implement Scannable Bool;
\end{lstlisting}

\subsection{IntN 类型}

\lstinline!Prelude! 中定义了几个定长的整数类型,用来扩展内置的 \lstinline!Int! 类型。它们包括 \lstinline!Int8!、\lstinline!Int16!、\lstinline!Int32! 和 \lstinline!Int128!。如其名称所暗示的,它们的内存大小就是 \lstinline!8!、\lstinline!16!、\lstinline!32! 和 \lstinline!128! 比特。

\begin{lstlisting}
export const Int8 = intrinsic;
export const Int16 = intrinsic;
export const Int32 = intrinsic;
export alias Int64 = Int;
export const Int128 = intrinsic;
\end{lstlisting}

这些类型的实际实现是编译器决定的,不过显然它们可以对应汇编指令集中的特定整数类型。这些类型都实现了 \lstinline!Integral! 类族,因此它可以进行所有整数的算术操作。下面以 \lstinline!Int8! 为例:

\begin{lstlisting}
export const + : Int8 -> Int8 -> Int8 = intrinsic;
export const neg : Int8 -> Int8 = intrinsic;
export const - : Int8 -> Int8 -> Int8 = x -> y -> x + (neg y);
export const * : Int8 -> Int8 -> Int8 = intrinsic;
export const div : Int8 -> Int8 -> (Int8, Int8) = intrinsic;
export const / : Int8 -> Int8 -> Int8 = x -> y -> (x `div` y)[0];
export const mod : Int8 -> Int8 -> Int8 = x -> y -> (x `div` y)[1];

implement Integral Int8;
\end{lstlisting}

当然,它们也实现了 \lstinline!Printable! 和 \lstinline!Scannable! 类族。

\begin{lstlisting}
implement Printable Int8;
implement Scannable Int8;
\end{lstlisting}

同时,这些类型也定义了从 \lstinline!Int! 隐式转换过来的函数,因此我们可以通过整数字面量来初始化一个 \lstinline!Int8! 类型的变量。

\begin{lstlisting}
export const of = (: Type ? Int8, x : Int) -> intrinsic;

// main.asc
const main = func () -> do {
    auto x : Int8 = 42;
};
\end{lstlisting}



\subsection{Proxy 类型}

\lstinline!Prelude.Proxy! 辅助生成结构中的 \lstinline!@prop! 属性,因此所有需要使用该属性的库都需要导入这个组件。

\begin{lstlisting}
@ctor
export const Proxy = struct (
    target : ref infer T,
    get : ref T -> ref infer U,
    set : (ref T, U) -> ref U
);
\end{lstlisting}

带有 \lstinline!@prop! 属性的成员会让编译器生成返回 \lstinline!Prelude.Proxy! 类型的代理对象,随后再根据重载的 \lstinline!<-! 运算符来控制访问行为:

\begin{lstlisting}
export const <- = (proxy : ref mut Proxy[infer T, infer U]) 
    -> func (val : U) -> (proxy.set)(proxy.target, val);

export const <- = (val : U) 
    -> func (proxy : ref Proxy[infer T, infer U]) -> (proxy.get)(proxy.target);
\end{lstlisting}
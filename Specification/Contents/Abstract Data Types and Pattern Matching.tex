\begin{introduction}
	\item ADT 表达式
	\item 标签
    \item 模式声明
    \item 原子模式
    \item 元组模式
    \item 修饰模式
    \item 选择模式
    \item 约束声明
    \item 约束的合取与析取
\end{introduction}

\section{抽象数据类型}

\subsection{ADT 表达式}

AutoScript 中用 \textbf{抽象数据类型(Abstract Data Type, ADT)} 来表示抽象的结构化数据类型。其语法如下:

\begin{grammar}[ADT 表达式 \texttt{[adt-expr]}] \label{grm:adt-expression}
	\lstinline!data [tag-choices]!
\end{grammar}

其中的每个类型标签都有下面的语法:

\begin{grammar}[ADT 标签声明 \texttt{[adt-tag-decl]}] \label{adt-tag-declaration}
\begin{enumerate}
	\item \lstinline![tag-id] [var-decl-list]!
	\item \lstinline![tag-id]!
\end{enumerate}
\end{grammar}

其中第二种是不带任何参数的标签,可以理解为带一个 \lstinline!Void! 参数的标签。相邻的标签用 \lstinline!|! 运算符相隔,这和选择类型的语法相似(实际上两者的本质也是相同的)。下面是几个例子:

\begin{lstlisting}
const Bool = data (True | False);
const MaybeInt = data (Value Int | Null);
const Color = data (RGB (r ?? 0.0 : Real, g ?? 0.0 : Real, b ?? 0.0 : Real) | NoColor);

const main = func () -> do {
	auto b = True;
	print typeof(b);		// 输出 Bool
	print tagof(b);			// 输出 True
	
	auto i = Value 42;
	print typeof(i);		// 输出 MaybeInt
	print tagof(i);			// 输出 Value
	
	auto c = RGB [r = 0.4, b = 0.6];
    print typeof(c);		// 输出 Color
    print tagof(c);			// 输出 RGB
};
\end{lstlisting}

可以看到,ADT 是一个选择类型和结构类型的组合体:每个标签都可以声明为一个结构类型,然后 ADT 是至少两个标签组成的选择类型。和结构表达式类似地,上面的圆括号并非必须,但为了避免运算符优先级造成的混乱,我们习惯上使用圆括号。

\subsection{标签的本质}

从标签的用法我们可以看出,它类似于一个函数。但更本质来说,它是一个使用了 \lstinline!@ctor! 属性的结构对象。

\begin{lstlisting}
const MaybeInt = data (Value Int | Null);

// 编译器会生成下面的代码
@ctor(MaybeInt)
const Value = struct (
	int : Int
);
@ctor @eager
const Null = struct ();
\end{lstlisting}

这里的 \lstinline!@eager! 属性可以让函数会自动被调用,其要求函数接收 \lstinline!Void! 为参数。 \\

因此 ADT 是一系列结构对象组成的选择类型,后者的很多性质同样适用于 ADT,比如强制类型转换。上面使用的 \lstinline!@ctor! 带了一个参数,它相当于指定了构造函数生成对象的类型(不是默认的 \lstinline!Value! 而是指定的 \lstinline!MaybeInt!),这也是为什么我们直接通过 \lstinline!Value 42! 就可以构建一个 \lstinline!MaybeInt! 对象。

\begin{lstlisting}
const MaybeInt = data (Value Int | Null);

const main = funct () -> do {
	auto i = Value 42;
	print i as Value;		// 输出 Value 42
};
\end{lstlisting}

标签是一个运行期的性质,编译器无法保证在编译期得到对象的标签性质。我们可以通过 \lstinline!tagof! 对类型的性质进行动态指派。

\begin{lstlisting}
const MaybeInt = data (Value Int | Null);

const f = funct (x : MaybeInt) -> do {
	if (tagof(x) == Value) {
		print "Value is $(x as Value)";
	}
	else {
		print "Nothing";
	}
};
\end{lstlisting}

不过这样的可读性比较差。我们通常会使用模式匹配对 ADT 类型对象进行操作。

\subsection{选择类型的模式匹配}

选择类型,包括 ADT 结构可以通过模式匹配来读取其中的值。

\begin{lstlisting}
const MaybeInt = data (Value Int | Null);

const f = funct (x : MaybeInt) -> do {
	match (x)
		Value i -> print "Value is $(i)",
		Null -> print "Nothing";
};
\end{lstlisting}

上面我们用 \lstinline!Value x! 来匹配 \lstinline!MaybeInt! 作为 \lstinline!Value! 时的结构,并声明了一个变量 \lstinline!i! 绑定其中存的值。特别地,\lstinline!Null! 后面不加任何变量。 \\

这种语法可以直接用于普通的选择类型,这是因为选择类型中的标签正是其自身(以 \lstinline!Int | String! 为例,其类似于一个 \lstinline!data (Int Int | String String)!。

\begin{lstlisting}
const f = funct (x : Int | String) -> do {
	match (x)
		Int i -> print "An integer $(i)",
		String s -> print "A string $(s)";
};
\end{lstlisting}

配合上一章介绍的 \lstinline!guard! 表达式,我们可以简洁地操纵选择类型对象。

\begin{lstlisting}
const f = funct (x : Int | String) -> do {
	x <| (guard
		Int i -> print "An integer $(i)",
		String s -> print "A string $(s)");
};
\end{lstlisting}

在之后的小节中,我们将正式且详细地介绍 AutoScript 的模式匹配机制。


\section{模式声明}

\textbf{模式匹配(Pattern Matching)}是 AutoScript 中无处不在的语言机制。其最典型用在实体声明和函数声明。模式匹配顾名思义,需要存在一个\emph{模式}和一个待匹配的实体。在实体声明中,出现在 \lstinline!=! 左侧的是模式,右侧则是一个待匹配的实体。函数声明中,出现在 \lstinline!->! 左侧的是模式,右侧则是函数体;在函数被调用时,参数就成为了待匹配的实体。

\begin{grammar}[模式匹配]
\begin{enumerate}
    \item \lstinline![pattern] = [init-expr];!
    \item \lstinline![pattern] -> [body]!
\end{enumerate}
\end{grammar}


AutoScript 中的模式中,存在三种不同作用的结构,其一是用于声明变量的部分,也被称为\emph{声明器};其二是用于限制匹配内容的\emph{约束};其三是暗示匹配结构的\emph{模式声明}。后两者是模式匹配的重点:它判断待匹配实体是否拥有特定结构并满足给定约束,如果是则利用声明器将匹配的结构或子结构声明为变量。 \\

本节让我们先熟悉 AutoScript 中所有模式声明的语法。

\subsection{原子模式}

原子模式中,模式声明里不存在子模式,此时最多能声明一个变量。下面是原子模式中所有可能出现的语法:

\begin{grammar}[原子模式 \texttt{[atomic-pattern]}] \label{grm:atomic-pattern}
\begin{enumerate}
    \item \lstinline!otherwise!
    \item \texttt{()}
    \item \texttt{[constr] [default-decl]}
    \item \texttt{[storage-spec] [id] [default-decl]}
    \item \texttt{[storage-spec] [id] [default-decl] [constr]}
    \item \texttt{[storage-spec] [id] [default-decl] [type-decl]}
    \item \texttt{[storage-spec] [id] [default-decl] [type-decl] [constr]}
\end{enumerate}
\end{grammar}

其中 \lstinline![constr]! 指的是模式中的约束,让我们在下一节中再详细介绍。下面是对这些语法的逐个说明:

\begin{enumerate}
    \item 跳过类型推断 \footnote{包括 \lstinline!undefined! 的情形,因此 \lstinline!otherwise! 会造成函数是不严格的。\textbf{严格函数(Strict Function)}要求 \lstinline!undefined! 输入一定会导致 \lstinline!undefined! 输出。考虑 \lstinline!\otherwise -> 42!,这显然不是一个严格函数。这个特点和 Haskell 完全一样。不过我们可以类似地使用不加约束的原子模式,如 \lstinline!x! 来强制求值。},可以匹配任何实体,但此时不声明任何变量。
    \item 匹配空对象 \lstinline!()!,此时不能声明任何变量 \footnote{敏锐的读者可能发现了这里的不和谐之处:\lstinline!()! 是一个对象,但为什么只有它被单独列出来?其它的譬如 \lstinline!42! 不能直接出现么?实际上,它们可以以约束的形式出现,比如 \lstinline!? 42!,而 \lstinline!()! 的特殊性是为了语言简洁性妥协的:我们避免类似于 \lstinline!\?() -> ...! 的函数出现。}。
    \item 一个约束,不声明任何变量。
    \item 无约束的自动类型推断,可以匹配任何实体并声明一个变量。
    \item 带有约束的自动类型推断,会将待匹配的实体传给约束判断其返回结果是否为真,若是则声明一个变量。
    \item 无约束的显式类型声明,可以匹配任何实体并将其隐式转化为给定类型后声明一个变量。
    \item 带有约束的显式类型声明,会将待匹配的实体传给约束判断其返回结果是否为真,若是则将实体隐式转化为给定类型后声明一个变量。
\end{enumerate}

原子模式是最常见的模式声明,其主要出现在变量的声明语句和函数的参数类型声明 \footnote{实际上,变量的提前声明对应了第 6 种情形,这也是唯一允许的提前声明语法。}。

\begin{lstlisting}
const main = funct () -> do {
    auto x = 42;                    // 原子模式 auto x
    auto f = funct (x : Int) -> x;   // 原子模式 auto f 和 x : Int
};
\end{lstlisting}

原子模式中最为特殊的是 \lstinline!otherwise! 这个占位符字面量。从其性质来看,它是一个极强的“混子”,所有和它相关的判断都会返回真。让我们尝试在其它地方使用 \lstinline!otherwise!。

\begin{minipage}[c]{0.95\textwidth}
\vspace{1.0em}
\begin{lstlisting}
const main = funct () -> do {
    auto x : Int = otherwise;		// 编译错误,otherwise 不能作为常规表达式求值
    auto otherwise = 42;			// 无事发生,没有真的 otherwise 作为变量被声明
    print otherwise;				// 编译错误,otherwise 不能作为常规表达式求值
};
\end{lstlisting}
\end{minipage}

从上面的例子可以看出, \lstinline!otherwise! 仅仅是模式声明中一个特殊的标识符,代表一个默认的匹配情形 \footnote{可以类比 C++26 中的占位符 \texttt{\_}。}。


\subsection{元组模式}

元组模式用于匹配一个形成元组结构的一系列模式。

\begin{grammar}[元组模式 \texttt{[tuple-pattern]}] \label{grm:tuple-pattern}
\begin{enumerate}
    \item \lstinline![pattern-list]!
    \item \lstinline!...[pattern]!
\end{enumerate}
\end{grammar}

这里 \lstinline![pattern-list]! 指的是由逗号分隔的多个 \lstinline![pattern]!。

\begin{enumerate}
    \item 完全匹配的元组模式,将元组中特定元素一对一匹配出来。
    \item 模糊匹配的元组模式,将元组匹配为一个\textbf{包(Pack)}。包中声明的所有变量都将带有\emph{可展开性},只需利用包展开 运算符 \lstinline!...! 即可将其展开为一个元组。
\end{enumerate}

\begin{lstlisting}
const main = funct () -> do {
    auto (x, ...xs) = (1, 2, 3);    // 元组模式 (x, ...xs),其中包含了另一个元组模式 ...xs
    auto ...ys = ...xs;             // 元组模式 ...ys,右侧通过 ... 运算符将包展开为元组
    print ...ys;                    // 输出 2, 3
};
\end{lstlisting}

我们有必要详述包的特性。首先,包的类型并不是一个元组,而是一个定义为内部实现的库类型 \lstinline!Prelude.Pack!。如上面所示,可以通过 \lstinline!...! 运算符将一个包转换为一个元组。不仅如此,对包进行的任何其它运算都会被视为对包中每个实体的运算。

\begin{lstlisting}
const main = funct () -> do {
	auto tup = (1, 2, 3);
	auto ...xs = mut tup;
	xs <- xs - 2;				// 对包中每个元素 x 都进行 x <- x - 2
	print xs;					// 输出 -1, 0, 1
};
\end{lstlisting}

此外,包还支持\emph{折叠表达式}。参考下面的说明:

\begin{itemize}
	\item \lstinline!(pack @ ...)!:左折叠,即展开为 \lstinline!(...((arg1, arg2) @ arg3) @ ...)!。
	\item \lstinline!(... @ pack)!:右折叠,即展开为 \lstinline!(arg1 @ (arg2 @ (arg3 @ ...)...)...)!
\end{itemize}

通过这种方式,我们可以轻松地求出一个数组中元素的和:

\begin{lstlisting}
const main = funct () -> do {
	auto arr = (1, 2, 3);
	print (arr + ...);		// 输出 6
};
\end{lstlisting}

或者依次处理一个包中的元素:

\begin{lstlisting}
const print_all = funct (...args) -> do {
	auto f = guard
		(x : Int) -> "This is an integer",
		(x : Real) -> "This is a real number",
		(x : String) -> "This is a string",
		otherwise -> "This is something else";
	(f(args), ...);
};
\end{lstlisting}

这里其实用到了 AutoScript 中的一个规则:在元组中每个元素会按照从左到右的顺序依次求值。

\subsection{修饰模式}

类型修饰符可以用来形成模式,包括 \lstinline!atom!、\lstinline!dyn!、\lstinline!mut!、\lstinline!ref!。

\begin{grammar}[修饰模式 \texttt{[qualifier-pattern]}] \label{grm:qualifier-pattern}
\begin{enumerate}
    \item \lstinline!atom [pattern]!
    \item \lstinline!dyn [pattern]!
    \item \lstinline!mut [pattern]!
    \item \lstinline!ref [pattern]!
\end{enumerate}
\end{grammar}

\begin{enumerate}
    \item 匹配一个原子实体,将其原型以别名方式取出来并通过 \lstinline![pattern]! 匹配。
    \item 匹配一个动态实体,将其原型以别名方式取出并通过 \lstinline![pattern]! 匹配。注意这里因为没有给出具体的实例类型,所以 \lstinline![pattern]! 可能被多种实例类型匹配。
    \item 匹配一个可变实体,将其原型以别名方式取出来并通过 \lstinline![pattern]! 匹配。
    \item 匹配一个引用实体,将其引用目标以别名方式取出并通过 \lstinline![pattern]! 匹配。
\end{enumerate}

模式到子模式的过程总是以别名方式构建的,这可以保证模式匹配中最多生成一个新的对象,避免内存的浪费。如果依然需要将父模式声明为变量,可以在 \lstinline![pattern]! 的最前面加上一个整体声明:

\begin{grammar}[整体声明 \texttt{[total-decl]}] \label{grm:total-declaration}
    \lstinline![id]~[pattern]!
\end{grammar}

下面是一些例子:

\begin{lstlisting}
const main = funct () -> do {
    auto x = mut 42;
    auto y~(mut n) = x;     // 修饰模式 mut n,并使用了整体声明 y
    print typeof(y);        // 输出 mut Int
    print typeof(n);        // 输出 Int
};
\end{lstlisting}

总体声明出来的变量和它进一步匹配的模式声明之间是别名的关系,比如上面的例子中,如果我们修改了 \lstinline!y! 的值,会影响 \lstinline!n! 的内容。不过不要因此弄混了模式声明和待匹配对象之间的关系:等号左边是右侧的复制,因此虽然

\subsection{可选修饰模式}

上节介绍的修饰模式有一个变种,即可选修饰模式,它使用 \lstinline!atom?!、\lstinline!dyn?!、\lstinline!mut?! 和 \lstinline!ref?! 来选择匹配一个可能出现的修饰符。

\begin{grammar}[可选修饰模式 \texttt{optional-qualifier-pattern}] \label{grm:optional-qualifier-pattern}
\begin{enumerate}
	\item \lstinline!atom? [pattern]!
	\item \lstinline!dyn? [pattern]!
	\item \lstinline!mut? [pattern]!
	\item \lstinline!ref? [pattern]!
\end{enumerate}
\end{grammar}

这种模式可以灵活地匹配任何表达式,且去除这个修饰符。

\begin{minipage}[c]{0.95\textwidth}
\vspace{1.0em}
\begin{lstlisting}
const main = funct () -> do {
    auto x = 42;
    auto y = mut 42;
    auto mut? z = x;
    auto mut? w = y;
};
\end{lstlisting}
\end{minipage}

这种模式和使用 \lstinline!demut! 等函数相比,更加偏向声明式;如果有命令式的代码习惯,使用 \lstinline!demut! 也无妨。下面是一个对比:

\begin{minipage}[c]{0.95\textwidth}
\vspace{1.0em}
\begin{lstlisting}
const f = funct (input : mut? Int) -> do {
    auto mut? x = input;
    auto y = demut input;
};
\end{lstlisting}
\end{minipage}

\subsection{选择模式}

正如上一节中我们介绍的,选择类型和抽象数据类型可以通过选择模式匹配。

\begin{grammar}[选择模式 \texttt{[choice-pattern]}] \label{grm:choice-pattern}
\begin{enumerate}
    \item \lstinline![tag-id] [pattern]!
    \item \lstinline![tag-id]!
\end{enumerate}
\end{grammar}

\begin{enumerate}
    \item 对应了有参数的标签,将标签包含的实体复制出来并通过 \lstinline![pattern]! 匹配。
    \item 对应了无参数的标签。需要注意这和 \lstinline![tag-id] ()! 是不同的:尽管它们都匹配了内存布局为 \lstinline!()! 的标签类型,但前者对应了拥有 \lstinline!@eager! 属性的构造函数,后者则没有该属性。
\end{enumerate}

\begin{minipage}[c]{0.95\textwidth}
\vspace{1.0em}
\begin{lstlisting}
const main = funct () -> do {
    scan x : Option[Int];
    match (x)
        Some e -> print e           // 选择模式 Some e
        Null -> print "Nothing";    // 选择模式 Null
};
\end{lstlisting}
\end{minipage}


\section{约束}

\subsection{约束声明}

在模式匹配中,我们会使用\textbf{约束(Constraint)}来对待匹配实体进行额外限制。约束分为\emph{值约束}和\emph{类型约束},顾名思义前者是待匹配实体值的\emph{谓词},后者则是其类型的\emph{谓词}。所谓\textbf{谓词(Predicate)},指的是返回可隐式转换为 \lstinline!Bool! 类型的函数。

\begin{grammar}[约束 \texttt{[constr]}] \label{grm:constr}
\begin{enumerate}
    \item \lstinline!? [pred]!
    \item \lstinline!:? [pred]!
\end{enumerate}
\end{grammar}

上面的 \lstinline!?! 称为值约束提示符,\lstinline!:?! 称为类型约束提示符。约束在变量声明处可能产生错误:如果类型约束不被满足会产生编译期错误;值约束不被满足则是运行期异常。在函数参数中出现的约束则作为重载决议判断的标准,编译器会按照顺序依次判断是否满足所有模式和约束。

\begin{lstlisting}
const not_zero = funct (x : Int) -> x <> 0;
const not_int = funct (const T : Type) -> T <> Int;

const main = funct () -> do {
    auto x ? not_zero = 42;     // 运行期检查其是否大于零
    auto y :? not_int = 42;     // 编译期错误
};
\end{lstlisting}

类型约束不仅仅可以用常规的类型谓词,也可以使用类型、类族、模版 \footnote{有关类族和模版的内容我们会在高级知识的部分介绍。}。

\begin{lstlisting}
const main = funct () -> do {
    scan x : Option[Int];
    match (x)
        :? Int -> print "Just an integer"
        :? Option -> do {
            print "Could be null";
            match (x)
                Some y -> print y;
                Null -> print "Nothing";
        }
        :? Printable -> print "We can print this";
};
\end{lstlisting}

此时匹配的逻辑是:

\begin{itemize}
    \item 若是一个类型,则判断待匹配实体的类型是否为该类型。
    \item 若是一个类族,则判断待匹配实体的类型是否属于该类族。
    \item 若是一个模版,则判断待匹配实体的类型是否从这个模版实例化得到。
\end{itemize}

注意区分类型约束和显式类型声明的区别:前者使用 \lstinline!:?! 提示符,后者使用 \lstinline!:! 提示符;前者会对待匹配实体进行检查,当且仅当待匹配实体满足给定的类型约束时才会完成匹配,后者则是在匹配成功后,为引入的新名字声明类型(并尝试进行隐式类型转换)。前者可能会匹配失败,但除非出现在变量定义语句中,否则不会产生错误;后者可能会出现隐式类型转换失败,错误一定会在编译期产生。 \\

此外值得说明的是,类型约束中不能出现标签类型,这是因为它是一个运行期信息;我们需要记住它要通过选择模式声明来进行模式匹配。当然,我们可以用值约束判断它是否是特定标签,但无法由此得到其中的具体数据,随后还是需要使用选择模式声明。

\subsection{约束合取与析取}

约束的合取与析取操作是借助查询结构完成的。我们已经在第六章中见到了查询结构对结构类型对象的筛选功能。这个功能当然也可以延续到模式匹配的约束中。

\begin{lstlisting}
const main = funct () -> do {
	const gt = funct (x : Int) -> funct (y : Int) -> y > x;
	const lt = funct (x : Int) -> funct (y : Int) -> y < x;
	
	scan x : Int;
	match (x)
		[? gt(0)][? lt(42)] -> "Greater than 0 and less than 42",
		[? lt(-100), ? gt(100)] -> "Too small or too large",
		otherwise -> "Try a smaller interval";
};
\end{lstlisting}

如果需要析取一个合取式,可以利用嵌套的查询结构;合取析取式时,和正常使用没有差异。

\begin{lstlisting}
import Math;

const main = funct () -> do {
	const gt = funct (x : Int) -> funct (y : Int) -> y > x;
	const lt = funct (x : Int) -> funct (y : Int) -> y < x;
	const sq = funct (x : Int) -> do {
		return x == Math.floor(Math.sqrt(x)) as Int;
	};

	scan x : Int;
	match (x)
		[? gt(42), [? sq][? lt(42)]] -> "Greater than 42 or is perfect square less than 42",
		[? sq][? lt(42), ? gt(42)] -> "Perfect square not equal to 42",
		otherwise -> "Other matchs";
};
\end{lstlisting}

此外,回忆函数之间可以进行运算,\lstinline!f @ g! 相当于 \lstinline!funct (...args) -> f(...args) @ g(...args)!。因此,我们可以用 \lstinline!and! 和 \lstinline!or! 来连接多个约束。

\begin{lstlisting}
const not_zero = funct (x : Int) -> x <> 0;
const is_even = funct (x : Int) -> x `mod` 2 == 0;

const main = funct () -> do {
    auto x ? not_zero and is_even = 42;
};
\end{lstlisting}


\subsection{动态约束类型}

从约束出发,我们可以设计一些拥有动态约束的类型。可以在结构类型声明或其构造函数中设置值约束。

\begin{minipage}[c]{0.95\textwidth}
\vspace{1.0em}
\begin{lstlisting}
const not_zero = funct (x : Int) -> x <> 0;
const Point = struct (
    x : Real ? not_zero,
    y : Real ? not_zero
);

const main = funct () -> do {
    auto p : Point = (1.0, 0.0);	// 运行错误,约束未满足
};
\end{lstlisting}
\end{minipage}


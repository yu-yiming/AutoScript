任何编程语言的学习都从一个简单的程序开始,本章我们将尝试实现一个词频统计程序。

\begin{introduction}
    \item 变量的定义
    \item 主函数
    \item 类型
    \item 注释
    \item 标准输入和输出
    \item \lstinline!if! 语句
    \item \lstinline!while! 语句
    \item \lstinline!for! 语句
    \item 函数
    \item 运算符
    \item 元组
    \item 结构
\end{introduction}

\section{Hello World 程序}

让我们从最简单的 Hello World 程序来开始 AutoScript 的学习。任何 AutoScript 程序都需要至少有一个\emph{语句}。通常 AutoScript 要求代码中有一个 \lstinline!main! \emph{函数},在经过编译后系统会尝试调用这个函数,并执行其中的代码。

\begin{lstlisting}
const main = func () -> do {
    print "Hello, World!";
};
\end{lstlisting}

虽然只有短短三行程序,但其中包含的知识点需要我们反复巩固才能深入理解,这里让我们逐个部分简单地说明它们的含义和作用。

\begin{itemize}
    \item \lstinline!const!:是一个\emph{存储类型限定符},用来定义一个变量。特别地,\lstinline!const! 表示\emph{常量存储期},常用于函数和类型的定义,我们这里先不细究其含义。类似这样拥有特殊含义的“单词”被称为\emph{关键字}。

    \item \lstinline!main!:也就是我们的\textbf{主函数(Main Function)},它是一个\emph{变量}。在 AutoScript 中,几乎所有的概念都可以绑定到一个名字上,此时它就成为了一个变量;关键字的名字不能用于变量绑定。主函数是程序的入口,系统会以\emph{调用}主函数作为程序运行的开始。

    \item \lstinline!=!:是一个\textbf{分隔符(Delimiter)},用来控制代码的结构。这里的 \lstinline!=! 用来定义变量,也被称为\emph{初始化器提示符}。

    \item \lstinline!func!:是一个\emph{实体类别提示符},其提示后续实体拥有的\emph{类别}。\lstinline!func! 也是一个\emph{阐述关键字},我们可以用符号 \lstinline!\! 来代替这个关键字。

    \item \lstinline!()!:一个空的\emph{函数参数列表},这里用来声明函数的参数。对于我们上面的例子,不需要接受任何参数,因此圆括号中是空的。

    \item \lstinline!->!:是一个\textbf{运算符(Operator)}。在 AutoScript 中有许多这样的连接符号,其将程序中的一些结构串联起来形成特殊的含义。对于函数来说,参数列表和函数体之间通过 \lstinline!->! 串联就得到一个\textbf{函数表达式(Function Expression)}。

    \item \lstinline!do!:是一个关键字,用来声明一个 \lstinline!do! \emph{表达式},后者是函数体常见的形式。在函数被调用时,\lstinline!do! 表达式的\emph{语句块},即大括号包围的部分中所有语句会被依次执行。

    \item \lstinline!{...}!:是一个语句块。写在 \lstinline!do! 关键字后可以构成一个 \lstinline!do! 表达式。

    \item \lstinline!print!:是一个关键字,会将后面的内容打印在指定的输出窗口上。

    \item \lstinline!"..."!:是一个\emph{字符串},在 AutoScript 代码中,有必要区分名字和字符串:前者是一个关键字或变量,后者则是一份数据。AutoScript 中使用双引号来表示字符串。

    \item \lstinline!;!:是一个分隔符,用来分开两个语句。
\end{itemize}

上面的程序会经过 AutoScript 编译器 \lstinline!asc! 编译后生成一个可执行文件,执行后就会在命令行中输出“Hello, World!” 。不过,如果直接在命令行中运行 \lstinline!asci!,则可以通过下面的代码来达到相同的效果:

\begin{lstlisting}
print "Hello, World!";
\end{lstlisting}

这是因为 \lstinline!asci! 是一个 REPL 程序,我们输入的所有指令都会被当场解释为某个中间代码,并通过 \lstinline!asci! \emph{解释}运行。由于 REPL 中一些代码的行为和常规程序不同,因此我们在需要使用 REPL 运行代码的例子会特别指出;没有特别说明时,都应该默认时常规的需要经过编译的代码。


\section{输入和输出}

\textbf{输入输出(IO)}并不是一个简单的概念,但是它们是从开始学习编程就一定会接触的编程语言知识。AutoScript 将标准输入输出纳入了\emph{语言特性}的范畴中,即我们不需要导入标准库来使用常规的输入输出操作;具体来说,我们可以通过 \lstinline!print! 和 \lstinline!scan! 关键字来进行 IO 操作。

\begin{lstlisting}
const main = func () -> do {
    auto x : Int = undefined;
    scan x;
    print x;
};
\end{lstlisting}

这里出现了一个新的\emph{存储限定符} \lstinline!auto!,用来表示\emph{自动存储期};它常用于普通数据类型变量的声明(比如一个整数或字符串等)。和 \lstinline!main! 函数不同的是,我们没有在变量名字后直接使用 \lstinline!=! 分隔符来直接定义它,而是使用 \lstinline!:! 分隔符来声明它的\textbf{类型(Type)}。 \\

AutoScript 中所有变量都拥有类型,它描述了一个对象的内存布局和行为特点,从而有助于在编译时对代码的行为进行检测和优化。上面的例子中,我们让 \lstinline!x! 的类型为 \lstinline!Int!,并通过 \lstinline!undefined! 来作为它的初始值。程序运行到这里时,会为 \lstinline!x! 分配空间但不对其初始化;所有读取 \lstinline!undefined! 的行为都是未定义的 \footnote{\textbf{未定义行为(Undefined Behavior)}是 C/C++ 中臭名昭著的特性之一,不过有的时候它也非常合理。AutoScript 中我们既然都字面意义上暗示了 \lstinline!undefined! 相关行为是未定义的,因此(或许)它能警示到更多人。}。好在随后我们将其交给了 \lstinline!scan! 语句,后者会根据命令行中的输入来为 \lstinline!x! 赋值。 \\

AutoScript 提供了一个更加方便的语法来从标准输入读取内容。

\begin{lstlisting}
const main = func () -> do {
    scan x : Int;
    print x;
};
\end{lstlisting}

如上面所示,我们可以在 \lstinline!scan! 语句中直接声明一个变量,它会通过输入的内容来初始化。这样就不用担心在变量声明到使用 \lstinline!scan! 语句之间误用变量了。这个技巧不仅限于 \lstinline!scan! 语句中,不过碍于篇幅让我们先记住它在 \lstinline!scan! 语句中的便捷使用。


\section{空格和注释}

读者或许已经注意到,前面的示例代码中,我们采用了特定的格式,比如在 \lstinline!do! 表达式的语句块中会换行并缩进、函数体中的语句每个都占用一行、关键字和符号之间用空格分开等。实际上这只是一个编写习惯。AutoScript 的名字和符号之间不会受空格和换行的影响。因此,下面的代码也是合法的:

\begin{lstlisting}
const
    main = func
() -> do { print "Hello, World"
;}
\end{lstlisting}

当然,这绝不是我们推荐的写法。通常来讲,代码应该通过换行和缩进来提示其结构,建议使用本文中使用的风格。 \\

除了空格之外,AutoScript 也会忽略\textbf{注释(Comment)}的存在。一共有两种注释,其一是行内注释,可以出现在每一行代码的末尾;其二是行间注释,可以出现在任何空格允许出现的地方。

\begin{lstlisting}
// 主函数
const main = func /* 参数列表 */ () -> {
    /*
        这里什么都没有写
     */
};
\end{lstlisting}

注释在代码中比较常见,它用来解释抽象的程序行为。


\section{控制流}

程序通常是从前到后依次执行的。不过在一些情况下,我们需要选择地执行部分代码,或重复执行部分代码,此时需要一些\emph{控制流语句}来描述代码行为。本节中介绍其中最常用的三种。

\subsection{if 语句}

程序中,我们时常会根据某个条件来判断是否要执行一些语句,比如我们根据工资的多少来决定税收的比重。

\begin{lstlisting}
const main = func () -> do {
    scan income : Int;
    if (income < 2000) {
        print "不需要缴税";
    }
    if (income >= 2000 and income < 8000) {
        print "需要缴 5% 税";
    }
    if (income >= 8000 and income < 20000) {
        print "需要缴 10% 税";
    }
    if (income >= 20000) {
        print "需要缴 20% 税";
    }
};
\end{lstlisting}

可以看到,\lstinline!if! 语句包含了 \lstinline!if! 关键字、一个由圆括号包围的判断表达式,还有一个语句块。当判断表达式求值为真时,就会执行语句块中的代码。此外,上面用到了 \lstinline!<! 和 \lstinline!>=! 运算符,它们的含义和数学中的小于和大于等于相同,而逻辑运算符 \lstinline!and! 当两边同时成立时才判断为真。


\subsection{while 语句}

\lstinline!while! 语句会按照某个条件重复执行一串代码,参考下面的“猜数游戏”:

\begin{lstlisting}
const main = func () -> do {
    auto x = Math.randint(0, 100);
    print "Guess a number between 0 and 100 (excluded)";
    scan guessed : mut Int;

    while (guessed <> x) {
        if (guessed < x) {
            print "Try a bigger one!";
        }
        else {
            print "Try a smaller one!";
        }
        scan guessed;
    }
    print "Congrats! You've nailed it!";
};
\end{lstlisting}

和 \lstinline!if! 语句类似地,\lstinline!while! 语句由 \lstinline!while! 关键字、一个圆括号包围的表达式还有一个语句块组成。执行到这个语句时,会首先判断表达式是否为真,若是则执行一次语句块,随后再次进行判断。直到判断条件为假前,循环会被不断执行。此外,上面用到了 \lstinline!<>! 运算符,它判断两个数是否不相等。我们也用到了 \lstinline!else! 子句,它紧跟在一个 \lstinline!if! 语句后,如果后者的判断条件为假,则会进入 \lstinline!else! 子句的语句块中。 \\

上面的例子中值得特别说明的是,我们将 \lstinline!guessed! 声明为 \lstinline!mut Int! 类型,这里的 \lstinline!mut! 是一个\emph{类型修饰符},其声明这个类型的对象是可变的。AutoScript 中默认所有变量都是不可变的,除非其被 \lstinline!mut! 修饰。由于我们在每次循环中都通过 \lstinline!scan! 语句重新对其赋值,因此需要将器声明为可变的 \footnote{此前我们使用 \lstinline!scan! 初始化的变量并不需要声明为 \lstinline!mut!,这是因为可变性只在初始化之后才有意义。}。

\subsection{for 语句}

对于一个\emph{范围},我们可以通过 \lstinline!for! 语句遍历其中的元素。下面让我们以数组类型为例。

\begin{lstlisting}
const main = func () -> do {
    auto arr : Int[3] = (1, 2, 3);
    for (e : Int from arr) {
        print e;
    }
};
\end{lstlisting}

这里,\lstinline!Int[3]! 表示拥有三个 \lstinline!Int! 对象的数组类型,其初始化为三个连续的元素 \lstinline!1, 2, 3!。\lstinline!e : Int from arr! 是一个比较复杂的结构,这里可以理解成声明了变量 \lstinline!e : Int! 后依次从 \lstinline!arr! 中取出元素。


\section{函数和运算符}

本节让我们简单介绍一下\textbf{函数(Function)}的定义和功能。函数可以看作是一个\emph{子程序},它拟接受一系列\emph{参数},并对一个表达式求值。当我们\textbf{调用(Invoke)}一个函数时,会将调用处的\textbf{参数(Argument)}用于初始化函数参数列表中的变量,并执行函数体。

\begin{lstlisting}
const add = func (x : Int, y : Int) -> x + y;	// 函数返回 x + y 的值

const main = func () -> do {
    scan a : Int;
    scan b : Int;
    print add(a, b);
};
\end{lstlisting}

上面的程序中,\lstinline!add! 和 \lstinline!main! 都是函数。在 \lstinline!main! 函数体中我们调用了 \lstinline!add! 函数,其语法是函数名紧跟一个数对 \lstinline!(a, b)!。调用函数时需要保证和参数列表中参数个数和对应的类型相同 \footnote{后面我们会学习特殊的情况,但这里先认为参数必须严格遵守函数定义}。 \\

运算符则是一类特殊的函数,它以中缀的形式调用。但在定义运算符时,和定义函数时没有太大区别。

\begin{lstlisting}
const <+> = func (x : Int) -> func (y : Int) -> x^2 + y^2;

const main = func () -> do {
    print 1 <+> 2;
};
\end{lstlisting}

注意到这里我们让函数返回了一个函数。这没有任何问题,因为函数本身也是一个对象。所有类似于 \lstinline!* -> * -> *! 的函数都称为\emph{运算符},它需要连续进行两次调用才能得到一个确切的值。从这个角度来看,\lstinline!a @ b!(其中 \lstinline!@! 是一个运算符)实际上是 \lstinline!@ a b! 的\textbf{语法糖(Syntax Sugar)},编译器会在编译时将其转化为实际的形式。  \\

无论是函数还是运算符,它们在 AutoScript 中的地位都和普通的对象无异:我们可以将其作为函数参数传递。

\begin{lstlisting}
const transform_twice = func (f : Int -> Int, x : Int) -> f(f(x));
const operate_reverse = func (op : Int -> Int -> Int, x : Int, y : Int) -> op(y, x);
const <-> = func (x : Int, y : Int) -> x - 2*y;

const main = func () -> do {
	print transform_twice(func (x : Int) -> x^2, 2);		// 输出 16
	print operate_reverse(<->, 24, 42);						// 输出 -6
};
\end{lstlisting}

上面的类型 \lstinline!Int -> Int! 表示了接受一个整数并返回一个整数的函数类型,可以看到 \lstinline!->! 在类型中的位置对应了函数声明时 \lstinline!->! 的位置。


\section{元组和结构类型}

AutoScript 中可以利用\textbf{元组(Tuple)}类型将多个类型按顺序排列在一起。元组可以通过逗号运算符 \lstinline!,! 构建。

\begin{lstlisting}
const main = func () -> do {
    auto tup = 42, "abc";
    print tup[0];
    print tup[1];
};
\end{lstlisting}

逗号运算符会将两个对象并列存储在一起,并返回这个元组对象。出于良好的习惯,此后所有的元组都会用圆括号包围 \footnote{这是因为逗号运算符的优先级比较低,容易被其它运算符影响}。从上面的代码中可以看到,我们可以通过 \lstinline!t[N]! 的形式来取得元组的某个元素,这是一个特殊的调用语法,让我们在后续章节再详细介绍 \footnote{实际上,\lstinline!t(N)! 也完全可以得到相同的结果,但为了符合主流语言的习惯,我们会使用方括号来取元素。}。 \\

AutoScript 中,可以使用\textbf{结构(Structure)}来定义元组的模版;在结构中,我们可以为元组的每个元素命名。

\begin{lstlisting}
const Point = struct (x : Real, y : Real);

const main = func () -> do {
    auto p : Point = (1.0, 0.0);
    print p.x;					// 访问第一个元素
    print p.y;					// 访问第二个元素
};
\end{lstlisting}

可以看到,结构也可以声明为一个变量。在定义特定结构类型的变量时,需要显式写出他的类型。结构类型的变量可以通过特殊的调用语法来访问它的元素,比如上面的 \lstinline!p.x! 就是访问了 \lstinline!p! 中被命名为 \lstinline!x! 的元素。 \\

一些结构上适合定义\textbf{方法(Method)},所谓方法是绑定在特定对象上的函数。

\begin{lstlisting}
const Point = struct (x : Real, y : Real);
const reset = func (this p : ref mut Point) -> do {
    p.x <- 0.0;
    p.y <- 0.0;
};

const main = func () -> do {
    auto p : mut Point = (1.0, 0.0);
    p.reset();			// 相当于 reset(p)
    print p.x;          // 输出 0.0
};
\end{lstlisting}

上面通过用 \lstinline!this! 修饰参数来让函数能通过符号 \lstinline!.! 的方法调用,这也是一个语法糖。在看到 \lstinline!p.reset! 时,编译器会生成一个表达式,其可以通过空对象 \lstinline!()! 得到我们想要的结果。可以参考下面的代码。

\begin{lstlisting}
const Point = struct (x : Real, y : Real);
const reset = func (this p : ref mut Point) -> do {
	p.x <- 0.0;
	p.y <- 0.0;
};

// 编译器会生成类似下面的代码
const compiler_generated_reset = func (p : ref mut Point) -> func () -> do {
	p.x <- 0.0;
	p.y <- 0.0;
};

const main = func () -> do {
	auto p : mut Point = (1.0, 0.0);
	p.reset();
	
	// 编译器会生成类似下面的代码
	compiler_generated_reset(p)();
};
\end{lstlisting}


\section{标准库}

编写 AutoScript 程序时,标准库是非常有用的帮手。我们不需要从头手动编写许多工具,因为它们已经在标准库中存在了。 \\

下面我们罗列一些常用的标准库部分和它们的作用。

\begin{itemize}
    \item \lstinline!Prelude! 库:其中包含了许多 AutoScript 必须的组件,包括 \lstinline!Prelude.Bool! 类型、\lstinline!Prelude.Proxy! 类型等。这个库会默认被 \lstinline!import! 到每个 AutoScript 文件中并\emph{内联},即我们可以直接使用 \lstinline!Bool! 来表示 \lstinline!Prelude.Bool!。

    \item \lstinline!Math! 库:其中包含了常用的数学函数,如 \lstinline!Math.max!、\lstinline!Math.sin! 等。

    \item \lstinline!System! 库:其中包含了和系统底层交互的函数,包括一些和 C/C++ 接口交互的函数。

    \item \lstinline!Containers! 库,其中包含了许多常用的容器和数据结构。

    \item \lstinline!Meta! 库,其中包含了许多和反射相关的组件。
\end{itemize}

如果想要引入某个标准库,需要使用 \lstinline!import! 关键字。

\begin{lstlisting}
import Math;

const main = func () -> do {
    print Math.sin(42)^2 + Math.sin(42)^2;  // 输出 1
};
\end{lstlisting}

本章中我们暂时不对上面提到的标准库详细介绍;后续章节在用到其中的组件时再进行解释。

\section{其它特性}



\section{一个词频统计程序}

首先,我们需要指定文件中读取文本。

\begin{lstlisting}
import System.IO;

const read_from = func (filename : String) -> do {
    auto file = IO.open(filename, "r");
    if (file == undefined) {
        throw "Cannot open file.";
    }
    return file.text();
};
\end{lstlisting}

这个返回的文本拥有 \lstinline!String! 类型,因此我们可以利用 \lstinline!split! 方法来根据空格或标点符号将其分开。

\begin{lstlisting}
const get_words = func (this text : String) -> do {
    const delim = func (ch : String) -> is_space(ch) or is_punct(ch);
    return text.split(delim);
};
\end{lstlisting}

随后,利用 \lstinline!HashTable! 容器来统计每个单词出现的次数。

\begin{lstlisting}
const count_words = func (this words : Vector(String)) -> do {
    auto table : mut HashTable(String, Int) = default;
    for (w : String from words) {
        if (not table.contains(w)) {
            table[w] <- 0;
        }
        table[w] <- table[w] + 1;
    }
    return table;
};
\end{lstlisting}

随后,我们需要关于 \lstinline!HashTable! 键值对中的值进行排序。为此可以使用 \lstinline!PriorityQueue! 容器。

\begin{lstlisting}
const comparator = func (x : (String, Int), y : (String, Int)) 
    -> x[1] < y[1] or (x[1] == y[1] and x[2] < y[2]);

const sort_counts = func (this table : HashTable(String, Int)) -> do {
    auto queue : mut PriorityQueue((String, Int), comparator) = default;
    for (e : (String, Int) from table) {
        queue.push(e);
    }
    return queue;
};
\end{lstlisting}

最后,可以根据输入的数量来输出前几位词频的单词。

\begin{lstlisting}
const most_frequent = func (queue : PriorityQueue((String, Int), comparator), k : Int) -> do {
    auto words : mut Vector(String) = default;
    auto i : mut Int = 0;
    while (i < k and not queue.is_empty()) {
        words.append(queue.top()[0]);
        i <- i + 1;
    }
    return words;
};
\end{lstlisting}

现在利用上面所有的函数,我们可以写出下面的程序:

\begin{lstlisting}
const main = func () -> do {
    print "Input the text file to read: ";
    scan filename : String;
    print "Input the k for the most frequent k words: ";
    scan k : Int;

    auto words = read_from(filename)
                .get_words()
                .count_words()
                .sort_counts()
                .most_frequent(k);
    for (w : String from words) {
        print w;
        print "\n";
    }
};
\end{lstlisting}
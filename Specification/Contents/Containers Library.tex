\begin{introduction}
	\item 
\end{introduction}

\section{Containers 库概览}

\lstinline!Containers! 库中定义了常用的几种数据结构,如 \lstinline!Containers.ArrayList!、\lstinline!Containers.HashTable! 和 \lstinline!Containers.Map! 等。下面让我们用数据结构的形式来分类介绍。


\subsection{线性存储类型}

包括一些抽象上是线性存储的容器,它们的特点是内存开销相对较小,但是排序性比较差。

\subsubsection{Containers.List}

\lstinline!List! 是一个动态长度的类似于 \lstinline!Array! 的类型 \footnote{熟悉其它语言的读者可以联系 C++ 中的 \lstinline!vector! 与 Java 中的 \lstinline!ArrayList!。}。它的内部实现中会分配比实际元素个数更多的内存,这样可以在插入元素时有更多的空间,直到不够时再重分配内存。

\begin{lstlisting}
export const List = struct (
	private data : ref infer T,
	@prop(get, undefined) size : Int,
	@prop(get, undefined) capacity : Int
);
\end{lstlisting}

\lstinline!Vector! 可以通过 \lstinline!make_list! 函数来构建。

\begin{lstlisting}
export const make_list = func (
	size : Int,
	capacity : Int
	value : infer T
) -> do {
	auto list_size = size ?? 0;
	auto list_cap = capacity ?? size ?? 8;
	assert list_size < list_cap;
	assert typeof(T) != Unresolved;
	auto list_data = malloc(sizeof(Int) * capacity) as ref mut T;
	if (value != missing) {
		for (elem from list_data) {
			list_data <- value;
		}
	}
	return (list_size, list_cap, list_data) as List;
};
\end{lstlisting}

使用示例如下:

\begin{lstlisting}
const main = func () -> do {
	auto v1 = make_list(3, 10, 42);				// 构建了内存长度为 10,拥有三个 42 的 List
	auto v2 = make_list[size = 3, value = 42];	// 构建了拥有三个 42 的 List
};
\end{lstlisting}

下面是一些和 \lstinline!List! 相关的函数:

\begin{enumerate}
	\item \lstinline!at!:访问特定位置的元素,会首先进行边界检查。
	\begin{itemize}
		\item 参数:
		\begin{enumerate}
			\item \lstinline!this : ref mut? List(infer T)!:目标列表。
			\item \lstinline!i : Int!:用整数表示的一个位置。
		\end{enumerate}
		\item 返回类型:\lstinline!ref mut?!。
		\item 异常类型:\lstinline!BoundaryError!:访问下标不合法时抛出。
	\end{itemize}
	
	\item \lstinline!push!:向列表结尾插入一个元素。
	\begin{itemize}
		\item 参数:
		\begin{enumerate}
			\item \lstinline!this : ref mut List(infer T)!:目标列表。
			\item \lstinline!e : ref? T!:按值或引用传递的一个元素。
		\end{enumerate}
		\item 返回类型:\lstinline!Iterator(List(T))!,列表最后一个元素的迭代器。
		\item 异常类型:\lstinline!()!。
	\end{itemize}
	
	\item \lstinline!pop!:从列表结尾移除一个元素。
	\begin{itemize}
		\item 参数:
		\begin{enumerate}
			\item \lstinline!this : ref mut List(infer T)!:目标列表。
		\end{enumerate}
		\item 返回类型:\lstinline!T!:移除的元素。
		\item 异常类型:\lstinline!EmptyError!,目标列表长度为零时抛出。
	\end{itemize}
	
	\item \lstinline!insert!:在列表中指定位置处插入一个元素。
	\begin{itemize}
		\item 参数:
		\begin{enumerate}
			\item \lstinline!this : ref mut List(infer T)!:目标列表。
			\item \lstinline!pos : Iterator(List(T))!:指定的位置。
			\item \lstinline!e : ref? T!:按值或引用传递的一个元素。
		\end{enumerate}
		\item 返回类型:\lstinline!Iterator(List(T))!,插入位置的迭代器。
		\item 异常类型:\lstinline!BoundaryError!,传入迭代器不在列表中时抛出。
	\end{itemize}
	
	\item \lstinline!remove!:在列表中指定位置处移除一个元素。
	\begin{itemize}
		\item 参数:
		\begin{enumerate}
			\item \lstinline!this : ref mut List(infer T)!:目标列表。
			\item \lstinline!pos : Iterator(List(T))!:指定的位置。
		\end{enumerate}
		\item 返回类型:\lstinline!T!,移除的元素。
		\item 异常类型:\lstinline!EmptyError!,目标列表长度为零时抛出。
	\end{itemize}
	
	\item \lstinline!reserve!:让列表保证给定的容量。
	\begin{itemize}
		\item 参数:
		\begin{enumerate}
			\item \lstinline!this : ref mut List(infer T)!:目标列表。
			\item \lstinline!cap : Int!:指定的容量大小。
		\end{enumerate}
		\item 返回类型:\lstinline!Int!,当前容量的大小。
		\item 异常类型:\lstinline!()!。
	\end{itemize}
	
	\item \lstinline!clear!:清除列表中所有元素。
	\begin{itemize}
		\item 参数:
		\begin{enumerate}
			\item \lstinline!this : ref mut List(infer T)!:目标列表。
		\end{enumerate}
		\item 返回类型:\lstinline!()!。
		\item 异常类型:\lstinline!()!。
	\end{itemize}
	
	\item \lstinline!resize!:修改列表大小为指定大小;如果小于当前元素个数则会截去所有多余的元素,大于当前元素个数则会用给定的值初始化新的元素。
	\begin{itemize}
		\item 参数:
		\begin{enumerate}
			\item \lstinline!this : ref mut List(infer T)!:目标列表。
			\item \lstinline!n : Int!:修改后列表的元素个数。
			\item \lstinline!e : ref? T!:新的元素的统一值,以值或引用传递。
		\end{enumerate}
		\item 返回类型:\lstinline!()!。
		\item 异常类型:\lstinline!()!。
	\end{itemize}
	
	\item \lstinline!shrink!:修改列表容量为当前元素个数,可能会发生内存重分配。
	\begin{itemize}
		\item 参数:
		\begin{enumerate}
			\item \lstinline!this : ref mut List(infer T)!:目标列表。
		\end{enumerate}
		\item 返回类型:\lstinline!()!。
		\item 异常类型:\lstinline!()!。
	\end{itemize}
	
\end{enumerate}